\section*{Задание 1}
\subsection*{Формулировка}
В транспьютерной матрице $6*6$, в каждом узле которой находится один процесс, необходимо выполнить операцию редукции MPI\_MAXLOC, определить глобальный максимум и соответствующих ему индексов.
Каждый процесс предоставляет свое значение и свой номер в группе. Для всех процессов операция редукции должна возвратить значение максимума и номер первого процесса с этим значением.
Реализовать программу, моделирующую выполнение данной операции на транспьютерной матрице при помощи пересылок MPI типа точка-точка.
Оценить сколько времени потребуется для выполнения операции редукции, если все процессы выдали эту операцию редукции одновременно. Время старта равно 100, время передачи байта равно 1 ($T_s=100$, $T_b=1$). Процессорные операции, включая чтение из памяти и запись в память, считаются бесконечно быстрыми.
\subsection*{Решение}
В рамках решения был разработан следующий алгоритм, состоящий из 6 этапов:
\subsubsection*{Алгоритм}
\begin{enumerate}
    \item 123
    \item 123
    \item 123
    \item 123
    \item 123
    \item 123
\end{enumerate}
Видно, что первые два этапа, а так же этапы 4 и 5 имеют схожую структуру, а, значит, их сожно выделить в две функции.
\begin{enumerate}
    \item \mintinline{c}{void collide_rows()} - для схлопывания двух крайних строк в строки, ближние к центру.
    \item \mintinline{c}{void compress_row()} - для схлопывания двух крайних столбцов в столбцы, ближние к центру.
\end{enumerate}

\subsubsection*{Оценка сложности алгоритма}
Сложность алгоритма оценивается формулой
$$
time = num\_steps \cdot \left( T_s + n \cdot T_b \right)
$$
где $num\_steps$ - количество шагов алгоритма, $n$ - размер сообщения, $T_s$ - время старта, $T_b$ - время передачи байта. В нашем случае $num\_steps = 6$, $T_s = 100$, $T_b = 1$. Размер сообщения (при предположении что \mintinline{c}{sizeof(int) == 4}) равен 12, так как помимо самого числа надо передать и координаты, что два числа. Тогда
$$
time = 6 \cdot \left( 100 + 12 \cdot 1 \right) = 600 + 72 = 672
$$

\subsubsection*{Текст программы}
\inputminted{c}{../task-1/main.c}